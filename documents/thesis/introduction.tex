
\chapter{Einleitung}
\label{einleitung}
\pagenumbering{arabic} % ab jetzt arabische Nummerierung

\textbf{EDIT: Die Arbeit wurde mit \LaTeX geschrieben und die ausgegebene PDF in eine Microsoft Word Datei, für die korrektur, konvertiert worden. Formattierungsfehler können auftreten!}
\vspace{1cm}

Im Rahman dieser Bachelorarbeit soll die Forschungsfrage geklärt werden, ob und mit welcher Umsetzung, eine Graduierung von Fazialisparesen durch Anwendung von Methoden des Maschinellen Lernens durchführbar ist. Für die Graduierung wird die House-Brackmann Skala angewendet, die als Standartmittel für die Einteilung gilt. Das Thema ist von hoher Relevanz. Es existiert zum jetzigen Zeitpunkt kein funktionsfähiges System, dass den Grad der Fazialisparese, aus gegebenen Bildern, unabhängig und korrekt feststellen kann.

Als Datensätze für den empirischen Teil der Arbeit stehen Patientenbilder zu Verfügung, die jeweils 9 Bilder mit unterschiedlichen Posen beinhalten. Diese wurden vom Universitätsklinikum Regensburg bereitgestellt. Zum Klassenausgleich des Datensatzes wird Oversampling genutzt welche eine Gleichverteilung, aus den bereitgestellten Datensätze, erzeugen soll. Das Verhalten der Genauigkeit mit und ohne Oversampling soll kritisch bewertet werden, ob bessere Ergebnisse so erzielt werden können.

Es werden zwei verschiedene Ansätze vorgestellt anhand dessen eine Graduierung durchgeführt werden kann. Dazu werden die verschiedenen Faktoren der House-Brackmann Skala als einzelne Module betrachtet, die in ihrem Bereich als Expertensysteme gelten. In diesen Modulen sind jeweils Neuronale Netze enthalten, die jeweils die Klassifikation der Module und ihren zugeordneten Klassen durchführen soll. Unter der Benutzung von Markerpunkten werden die 9 Bilder in Teilregionen zerlegt, die als Eingabe in die Module dienen. Durch das Verwenden eines Automaten oder der Zeilensummenoperation werden die einzelnen Klassen der vier Module fusioniert, um den Ausgabegrad nach House-Brackmann zu bestimmen. Eine weitere Möglichkeit wird betrachtet, welchesden Grad der Fazialisparese unter der Anwendung der House-Brackmann Skala direkt den Grad von I bis VI zu ermitteln versucht, ohne den Umweg über die Module zu nehmen. Dies dient als Referenz zum Vergleich zwischen Direkt und Modulform.

Verschiedene Verfahren zur Abarbeitung der 9 Bilder der Datensätze von Patient*in werden vorgestellt. Im sequenziellen Verfahren werden die 9 Bilder der/die Patient*innen hintereinander in die Neuronalen Netze hineingegeben. Mit der Anwendung von Early Fusion werden diese vorab konkateniert und als gemeinsames Paket in die Neuronalen Netze hineingegeben. Late Fusion hingegen bekommt jedes Bild für jedes Modul sein eigenes Netz. Experimentell sollen die verschiedenen Verfahren, mit und ohne Oversampling zum Klassenausgleich und jeweils mit dem zwei Ansätzen der Graduierung nach House-Brackmmann, durchgeführt werden. Die verschiedenen Experimente werden im späteren Verlauf kritisch bewertet und verglichen.

Auch wird kurz angerissen, in welcher Form Caching zur Beschleunigung der Trainingsphase zu Anwendung kommen kann. Dazu werden einmal eine externe Datenbank verwendet und die im Sourcecode verwendete Datenstruktur Least Recently Used Cache, die intern die benötigten Daten aufbewahrt. Den Beschleunigungsfaktor soll experimental analysiert werden.

Zuletzt werden die Vor- und Nachteile der verschiedenen Methoden und Verfahren kurz angesehen, kritisch bewertet, mit anderen schon existierenden Systemen verglichen und eingeordnet. Des Weiteren wird ein kurzer Ausblick auf die weiteren Möglichkeiten zur Weiterentwicklung und Anwendung der entstandenen Experimente gegeben.
