
\chapter{Einleitung}
\label{einleitung}
\pagenumbering{arabic} % ab jetzt arabische Nummerierung

Im Rahmen dieser Bachelorarbeit soll die Forschungsfrage geklärt werden, ob und mit welcher Umsetzung eine Graduierung von Fazialisparesen durch Anwendung von Methoden des Maschinellen Lernens durchführbar ist. Für die Graduierung wird die House-Brackmann Skala angewendet, die als Standartmittel zur Einteilung gilt. Das Thema ist von hoher Relevanz. Zum jetzigen Zeitpunkt existiert kein funktionsfähiges System, das den Grad der Fazialisparese aus gegebenen Bildern unabhängig und korrekt, im Praxisalltag duch das Nutzen von Neuronalen Netzen, feststellen kann.

Als Datensätze für den empirischen Teil der Arbeit stehen Patientenbilder zur Verfü- gung, die jeweils neun Bilder mit unterschiedlichen Posen beinhalten. Diese wurden vom Universitätsklinikum Regensburg bereitgestellt. Zum Klassenausgleich des Datensatzes wird Oversampling genutzt, welches eine Gleichverteilung aus den bereitgestellten Datensätzen erzeugen soll. Das Verhalten der Genauigkeit mit und ohne Oversampling soll kritisch bewertet werden, ob so bessere Ergebnisse erzielt werden können.

Zwei verschiedene Ansätze werden vorgestellt, anhand derer eine Graduierung durchgeführt werden kann. Dazu werden die verschiedenen Faktoren der House-Brackmann Skala als einzelne Module betrachtet, die in ihrem Bereich als Expertensysteme gelten. Diese Module enthalten jeweils Neuronale Netze, die wiederum die Klassifikation innerhalb der Module und ihren zugeordneten Klassen durchführen sollen. Unter Benutzung von Markerpunkten werden die neun Bilder in Teilregionen zerlegt, die als Eingabe für die Module dienen. Durch das Verwenden eines Automaten oder der Zeilensummenoperation werden die einzelnen Klassen der vier Module fusioniert, um den Ausgabegrad nach House-Brackmann zu bestimmen. Eine weitere Möglichkeit wird betrachtet, welche der Fazialisparese unter der Anwendung der House-Brackmann Skala direkt den Grad von I bis VI zuzuordnen versucht, ohne den Umweg über die Module zu nehmen. Dies dient als Referenz zum Vergleich zwischen Direkt und Modulform.

Verschiedene Verfahren zur Abarbeitung der neun Bilder der Datensätze von Patient*in werden vorgestellt. Im sequenziellen Verfahren werden die neun Bilder der/die Patient*innen hintereinander in die Neuronalen Netze eingegeben. Mit der Anwendung von Early Fusion werden diese vorab konkateniert und als gemeinsames Paket in die Neuronalen Netze eingespeist. Bei der Late Fusion hingegen wird jedem Bild für jedes Modul ein eigenes Netz zugewiesen. Experimentell sollen die verschiedenen Verfahren mit und ohne Oversampling zum Klassenausgleich und jeweils mit beiden Ansätzen der Graduierung nach House-Brackmann durchgeführt werden. Die verschiedenen Experimente werden im späteren Verlauf kritisch bewertet und verglichen.

Auch wird kurz angerissen, in welcher Form Caching zur Beschleunigung der Trainingsphase der Neuronalen Netze zur Anwendung kommen kann. Dazu werden eine externe Datenbank, deren ihren Speicher sich auf einer Festplatte (SSD, HDD) und die im Sourcecode verwendete Datenstruktur Least Recently Used Cache verwendet, die intern, die benötigten Daten im RAM des Systems, aufbewahrt. Der Beschleunigungsfaktor soll experimental analysiert werden und welche Vor- und Nachteile die Cachingmethoden besitzen.

Zuletzt werden die Vor- und Nachteile der verschiedenen Methoden und Verfahren kurz betrachtet, kritisch bewertet, mit anderen, bereits existierenden Systemen verglichen und eingeordnet. Des Weiteren wird ein kurzer Ausblick auf die Möglichkeiten zur Weiterentwicklung und Anwendung der entstandenen Experimente gegeben. Dazu zählen Verschlüsselung der Daten, einen kurzen Einblick der Vor- und Nachteile der Architekturaufbauten Thick- und Thin-Client und die Informationsfusion mit einer weiteren Skala.
