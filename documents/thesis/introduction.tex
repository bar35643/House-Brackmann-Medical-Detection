
\chapter{Einleitung}
\label{einleitung}
\pagenumbering{arabic} % ab jetzt arabische Nummerierung

Dieses \LaTeX-Template soll die Erstellung von Bachelor- und Masterarbeiten erleichtern.
Bei externen Arbeiten ist das ReMIC Logo durch das Fakultät-IM Logo zu ersetzen und durch das Firmenlogo auf der rechten Seite zu ergänzen.
Außerdem sollte neben den Gutachtern der Betreuer im Unternehmen namentlich auf der Titelseite zu erwähnen.

Es wurde versucht, in diesem Dokument die wichtigsten Elemente einer Abschlussarbeit beispielhaft zu verwenden.
Dazu gehören Listen, Tabellen, Bilder, wörtliche und nicht-wörtliche Zitate, zitierweisen von Zeitschriftenartikeln, Büchern, Abschlussarbeiten und Webseiten, Gleichungen und Referenzierungen von Kapiteln und Graphiken.

\paragraph{Vollständigkeit}
Die Verwendung der Elemente erhebt keinen Anspruch auf Vollständigkeit.
Allerdings ist das Internet voll von Hilfen für \LaTeX, deshalb werden Sie keine Probleme haben, wenn Sie z.B. nicht mehr wissen, wie $\gamma$ im Mathematik-Modus geschrieben wird, das herauszufinden.

\paragraph{Hauptkapitel}
In diesem Dokument wird jedes Hauptkapitel auf der rechten Seite eines doppelseitigen Drucks begonnen.
Dadurch bleibt auch schon mal eine Seite leer, das macht aber nichts und erleichtert in der gebundenen Form das Lesen.
Im Allgemeinen sollten sich in irgendeiner Form die Kapitel \emph{ Einleitung}, \emph{Methode}, \emph{Ergebnisse}, \emph{Diskussion} und \emph{ Ausblick}.
Im Inhaltsverzeichnis sollten nur Kapitel bis zur Tiefe \emph{ Subsection}, also mit 3 Nummern, erscheinen.

Dieses Dokument ist als Hilfe gedacht, damit Sie schnell mit dem Tool \LaTeX~zurecht kommen,
damit Sie die wichtigsten Elemente einer sehr guten Abschlussarbeit zum Nachlesen haben und damit Sie meine \textbf{notenrelevanten Vorgaben} in schriftlicher Form vorliegen haben.
Mit diesem Template sollen nicht alle Abschlussarbeiten völlig gleich aussehen, insbesondere nicht sprachlich.
Entwickeln Sie Ihren eigenen Stil, seien Sie kreativ und bleiben Sie sich treu. Innerhalb dieser Vorlage gibt es jede Menge Spielraum.











% \begin{tikzpicture}[->,>=stealth',shorten >=1pt,auto,node distance=2.8cm,
%                     semithick]
%   \tikzstyle{every state}=[fill=white,draw=black,text=black]
%
%   \node[initial,state] (A)                    {$q_a$};
%   \node[state]         (B) [above right of=A] {$q_b$};
%   \node[state]         (D) [below right of=A] {$q_d$};
%   \node[state]         (C) [below right of=B] {$q_c$};
%   \node[state]         (E) [below of=D]       {$q_e$};
%
%   \path (A) edge              node {0,1,L} (B)
%             edge              node {1,1,R} (C)
%         (B) edge [loop above] node {1,1,L} (B)
%             edge              node {0,1,L} (C)
%         (C) edge              node {0,1,L} (D)
%             edge [bend left]  node {1,0,R} (E)
%         (D) edge [loop below] node {1,1,R} (D)
%             edge              node {0,1,R} (A)
%         (E) edge [bend left]  node {1,0,R} (A);
% \end{tikzpicture}

\definecolor{mycolor}{RGB}{50,180,255}

\begin{tikzpicture}[->,>=stealth',shorten >=1pt,auto,node distance=2.8cm,
                    semithick]
  \tikzstyle{every state}=[fill=mycolor,draw=black,text=black]

  \node[initial,state] (A)                    {$I$};
  \node[state]         (B) [right of=A]       {$II$};
  \node[state]         (C) [below of=B]       {$III$};
  \node[state]         (D) [below of=A]       {$IV$};
  \node[state]         (E) [below of=C]       {$V$};
  \node[state]         (F) [below of=D]       {$VI$};

  \path (A) edge              node {mouth=1} (B)
            edge [loop above] node {mouth=0} (A)
            edge              node [text width=2.3cm,midway,left]{mouth=2 or forehead=2}   (D)
        (B) edge              node {forehead=1} (C)
            edge [loop above] node {forehead=0} (A)
        (C) edge              node {eye=1} (D)
            edge [loop right] node {eye=0} (C)
        (D) edge              node [right]{symmetry=1} (E)
            edge [loop left]  node {symmetry=0} (D)
            edge              node [text width=2.3cm,midway,left]{symmetry=2 or mouth=3} (F);
\end{tikzpicture}
