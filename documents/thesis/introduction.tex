
\chapter{Einleitung}
\label{einleitung}
\pagenumbering{arabic} % ab jetzt arabische Nummerierung

Im Rahman dieser Bachelorarbeit soll die Forschungsfrage geklärt werden, ob und mit welcher Umsetzung, eine Grauierung von Fazialisparesen durch Anwendung von Methoden des Maschinellen Lernens durchfürbar ist. Für die Graduierung wird die House-Brackmann Skala angewentet, die als Standartmittel für die Einteilung gilt. Als Datensätze stehen Patientenbilder zu Verfügung, die jeweils 9 Bilder mit unterschiedlichen Posen beinhalten. Oversampling zum Kassenausgleich kommt zum Einsatz, um den ungleichen Datensatz auszugleichen.
Es werden zwei verschiedene Ansätze vorgestellt anhand dessen eine Graduierung durchgeführt werden kann. Dazu werden die verschiedenen Faktoren der House-Brackmann Skala als einzelne Module betrachtet, die im ihren Bereich als Expertensysteme gelten. In diesen Modulen sind jeweils Neuronale Netze enthalten, die jeweils die Klassifikation der Module und ihren zugeordneten Klassen durchführen soll. Unter der Benutzung von Markerpunkten werden die 9 Bilder in Teilregionen zerlegt, die als Eingabe in die Module dienen. Anhand eines Automaten oder der Zeilensummenoperation werden die einzelnen Klassen der vier Module fusioniert, um den Ausgabegrad nach House-Brackmann zu bestimmen. Eine andere Möglichkeit ist es, den Grad der Fazialisparese unter Anwendung der House-Brackmann Skala direkt zu ermitteln, ohne den Umweg über die Module zu nehmen. Dies dient als Referenz zum Vergleich zwischen Direkt und Modulform.
Verschiedene Verfahren zur Abarbeitung der 9 Bilder der Datensätze von Patient*in werden vorgestellt. Im sequenziellen Verfahren werden die 9 Bilder der/die Patient*in hintereinander in die Neuronalen Netze hineingegeben. Mit der Anwendung von Early Fusion werden diese vorab konkateniert und als gemeinsames Paket in die Neuronalen Netze gesteckt. Late Fusion hingegen bekommt jedes Bild für jedes Modul sein eigenes Netz. Experimentell sollen die verschiedenen Verfahren, mit und ohne Oversampling zum Klassenausgleich und jeweils mit dem zwei Ansätzen der Graduierung nach House-Brackmmann, durchgeführt werden. Die Verschiedenen Experimente werden im späteren Verlauf kritisch bewertet und verglichen.
Auch wird kurz angerissen, in welcher Form Caching zur Beschleunigung der Trainingsphase zu Anwendung kommen kann. Dazu werden einmal eine Externe Datenbank verwentet und die im Sourcecode verwendete Datenstruktur Least Recently Used Cache, die intern die benötigten Daten aufbewahrt. Den Beschleunigungsfaktor soll experimental Analysiert werden.
Zuletzt werden die Vor- und Nachteile der verschiedenen Methoden und Verfahren kurz angsehen, kritisch bewertet, mit anderen schon existierende Systemen verglichen und eingeordnet. Desweiteren wird ein kurzer Ausblick auf die weitere Möglichkeiten zur Weiterentwichkung und Anwendung der entstandenen Sourcen gegeben.
