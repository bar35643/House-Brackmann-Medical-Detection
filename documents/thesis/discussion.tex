\chapter{Diskussion und Ausblick}\label{discussion}
\section{Vor- und Nachteile der direkten Ermittlung gegenüber dem Modulaufbau}\label{di_vs_module}
Die Vorteile der Modulform liegen in der unabhängigen Funktionalität der Module. Diese besitzen ihre jeweils zugeschnittenen Bildmaterialien für die entsprechenden Module. Durch die unabhängige Funktion ist ebenso sichergestellt, dass, falls ein Modul für eine Person eine nicht korrekte Klassifikation ausführt, die anderen Module nicht von ihm beeinflusst werden. Ein weiterer Vorteil ist, dass durch den hochmodularen Aufbau einzelne Module ausgegliedert und als Expertensysteme separat in anderen Projekten ohne weiteres Zutun verwendet werden können. Einzige Voraussetzung hierbei ist, dass diese Systeme eine korrekte Klassifizierung ausführen.

Problematisch ist allerdings, dass, falls die Einteilung der Regions of Interest durch das verwendete Framework für die einzelnen Module nicht korrekt funktioniert und somit falsche Bereiche ausgewählt werden, keine Korrektur oder Fehleranalyse erfolgen kann. Dadurch kann es zu fehlerhaften Klassifikationen kommen, die so nicht feststellbar sind. Des Weiteren ist von Nachteil, dass die Ermittlung des Grades aus vier Modulen besteht, die im Nachhinein zu dem Grad nach House-Brackmann fusioniert werden. Die Schwachstelle dabei ist, dass verschiedene Wege zur Bildung des Grades führen. Dabei ist die Verwendung des Automaten oder die Nutzung der Zeilensumme der freien Entscheidung überlassen und somit wieder subjektiv in der Meinung des/der jeweiligen Anwender*ins. Das wiederum ist für Patient*innen von Nachteil, da der Ausgabegrad objektiv sein soll.

Die Anwendung der direkten Ermittlung des Grades hingegen hat den Vorteil, dass keine Fusionierung durch Automaten, Zeilensumme oder anderes nötig ist. Diese Methode basiert nur auf der Eingabe der neun Bilder der/die Patient*innen und den als Ausgabe rückzugebenden Grad. Auch ist so keine Modulbildung nötig, welches das Vierfache an Zeit an der obigen dargestellten Methode benötigt. So können auch besser echtzeit- kritische Detektionen ausgeführt werden, welche für die Praxisanwendung vorteilhaft wären.

Nachteilhaft ist, dass anhand der direkten Ermittlung, falls ein Fehler in der Klassifikation stattfindet, einzig und allein durch das eine Neuronale Netz der direkten Ermittlung der Fehler verursacht werden kann. So wird umso wichtiger, dass das Training und die Validierung der Netze in einem Datensatz ausgeführt werden, welcher mehrere tausend Elemente beinhaltet. Die Sammlung einer solchen Anzahl an Patientendaten, welche ebenfalls noch Datenschutzkonform gesammelt werden müssen, ist eine weitere große Herausforderung.


\clearpage
\section{Vor- und Nachteile der verschiedenen Vorgehensweisen}\label{early_vs_late}
Die verschiedenen Betrachtungsweisen zur Fusion bzw. Abarbeitung der neun Bilder der/die Patient*innen haben ihre Vor- und Nachteile. Die sequenzielle Abarbeitung der Bilder ist zwar einfach in der Umsetzung. Jedoch ist denkbar, dass der Fortschritt der Neuronalen Netze zum Teil überschrieben und rückgängig gemacht wird. So kann zum Beispiel der Erfolg von Bild 1 durch die anderen acht Bilder zunichte gemacht werden. Sichtbar wird dies durch die sehr sprunghaften Veränderungen im F1-Graph und den erst verzögerten Anstieg bei der direkten Ermittlung.

Das Zerschneiden der Bilder in die Regions of Interest für die sequenzielle Füt- terung der Neuronalen Netze der Module hat keine positiven Auswirkungen auf eine Verbesserung einer korrekteren Klassifizierung. Da kein Anstieg, weder mit noch ohne Oversampling zum Klassenausgleich, ersichtlich ist, scheint sich die Annahme, dass sich die neun Ausschnitte der Bilder für jedes Modul gegenseitig zu überschreiben, zu bestätigen.

Late Fusion mit der Nutzung von Oversampling hat den Nachteil, dass die Detektion sehr viel Zeit in Anspruch nimmt. Trotz dessen, dass die Modulform insgesamt 36 Neuronale Netze (pro Bild eins und das für die vier Module) nutzt, ist, im Gegensatz zu der sequenziellen Verarbeitung, ein relativ konstantes Wachstum zu verzeichnen. Für die direkte Ermittlung beim sequenziellen Verfahren sowie beider (Direkt und Modulform) für Late Fusion hätten mehr als die 150 Epochs benötigt, weil der F1-Wert noch kein Plateau bzw. die Kurve kein Abfallen aufzeigt. Mit mehr Epochs hätten die Neuronalen Netze die Chance, einen Zuwachs des F1-Wertes zu erzielen.

Das beste Ergebnis wurde sowohl für die Modulform als auch für die direkte Detektion mit der Nutzung von Early Fusion erzeugt. Anzunehmen ist, da der F1-Wert nahezu perfekt durch Oversampling bzw. eins ohne Anwendung von Oversampling ist, dass zu wenige Patient*innen im Datensatz enthalten sind. Der Verdacht liegt nahe, trotz der Trennung in einen Trainings- und Validierungsdatensatz, dass das Neuronale Netz die konkatenierten Bilder auswendig gelernt hat. Auch kann es sein, dass die Augmentierung der Bilder zu schwach eingestellt ist. Dagegen spricht, da es sich um neun verschiedene Posen handelt, die für jede*n Patient*in vorhanden sein sollten, vorab konkateniert und dem Netz überlassen worden ist, den richtigen Schluss zur richtigen Klasse des Grades ziehen soll.

Für alle Vorgehensweisen, sequenzielle Anordnung, Early und Late Fusion, für die vorgestellten Methoden Modulfom und direkten Ermittlung ist die Menge enthaltener Patient*innen im Datensatz zu gering, um ein stichhaltiges aussagekräftiges Ergebnis der Experimente zu erhalten. So wie die F1-Graphen der Experimente darstellen, kann eine Tendenz erkannt werden, dass generell die Möglichkeit besteht, mit den genannten Methoden durch das Nutzen von Neuronalen Netzen eine Ermittlung des Grades nach House- Brackmann durchführbar ist. Für ein aussagekräftiges Ergebnis zur Genauigkeit und Richtigkeit der festgestellten Klassen müssen die Experimente wiederholt werden. Die Neuausführung der Experimente sollte mit einen deutlichen größeren Datensatz erfolgen.












\clearpage
\section{Zusammenfassung}\label{conclude}
Zusammenfassend wurden verschieden Experimente zu den Methoden, die House- Brackmann Tabelle als einzelne Module zu betrachten, und der direkten Ermittlung der Grade I - VI, welche die Einteilungsstufen der House-Brackmann Grade ist, durchgeführt. Dabei wurden Experimente zu den verschiedenen Vorgehensweisen zur Verarbeitung der neun Bilder vorgestellt bzw. durchgeführt. Die Vorgehensweisen sind die sequenzielle Anordnung, Early Fusion durch Konkatenation und Late Fusion, wobei jedes Bild ein eigenes Netz erhielt. Zum Ausgleich der nicht gleichverteilten Klassen der/die Patient*innen wurde Oversampling zum Klassenausgleich sowohl in der Modulform für die einzelnen Bestandteile der House-Brackmann Skala als auch in der direkten Ermittlung erfolgreich angewendet. Je nach Vorgehensweise hatte Oversampling eine andere Auswirkung auf die Schnelligkeit des Trainingserfolges, sichtbar durch den F1- Wert.

Der Vorteil der Modulform ist die unabhängige Nutzung in anderen medizinischen Bereichen. Dort kann, falls ein anderes Experiment oder eine andere Anwendung ein Modul z. B. Liedschluss als Kriterium benötigt, auch ohne die Ausführung der Detektion des Grades genutzt werden. So sind, durch den hoch modularen Aufbau alle Komponenten, Einteilung in die Regions of Interest, die Module sogar die Fusionierung des Grades einzeln oder auch im Gesamtpaket nutzbar. Nachteilhaft ist, dass die Modulform mehrere Neuronale Netze beinhaltet und deswegen wesentlich mehr Rechenzeit benötigt als die direkte Form der Ermittlung. Bewiesen wurde auch, dass es möglich ist, durch das aktive Benutzen von Caches mit den Implementierungsformen \ac{lru} und Datenbank, schneller die Trainingsphase der Neuronalen Netze durchlaufen zu lassen.

\begin{table}[!h]\vspace{3ex}\centering
  \begin{tabular}{l||c|}
  \textbf{Projekte}                           & \textbf{F1-Wert in \%} \\ \hline\hline
  Insu et al. \cite{6583060}                    & 89.23   \\
  Hyun et al. \cite{s151026756}                 & 87.11   \\
  Muhammad et al. \cite{sym10070242}            & 92.91   \\
  Ting et al. End-to-End \cite{detection_fp2}   & 83.41   \\
  Ting et al. Kaskadierend \cite{detection_fp2} & 95.97   \\ \hline
  Modulform Sequenziell (Unser)                 & 35.50   \\
  Modulform Early Fusion (Unser)                & 98.00   \\
  Modulform Late Fusion (Unser)                 & 92.70   \\
  Direkt Sequenziell (Unser)                    & 91.40   \\
  Direkt Early Fusion (Unser)                   & 100.00  \\
  Direkt Late Fusion (Unser)                    & 92.70   \\ \hline
  \end{tabular}
  \caption[Leistungsvergleiche der Graduierung der Fazialisparese von verschiedenen Projekten]{Leistungsvergleiche der Graduierung der Fazialisparese in Bezug auf den F1-Wert. Einordnung zu anderen Experimenten und Projekten. Dazu sind jeweils die besten der Modulform und der Direkten Ermittlung genommen worden \cite{detection_fp2}.}\label{cap:ref_tab}
\vspace{1ex}\end{table}\label{table:ref_tab}

\clearpage

Zur Einordnung der Qualität und Funktionalität der durchgeführten Experimente wurden diese mit anderen Projekten verglichen (siehe Tabelle \ref{cap:ref_tab}). Leider sind die Ergebnisse der Experimente nicht aussagekräftig genug, da zu wenig Patient*innen sowohl im Trainings- als auch im Validierungsdatensatz enthalten sind. Der Verdacht liegt nahe, dass die Neuronalen Netze bei der Anwendung von Early und Late Fusion sowie bei der sequenziellen Anordnung die Bilder auswendig gelernt oder die Augmentierung zu schwach eingestellt war. Der Vorteil, im Gegensatz zu den anderen Projekten, ist die Nutzung von neun verschiedenen Posen, wodurch theoretisch die Feststellung des Grades nach House-Brackmann leichter sein sollte.
