
\appendix
\chapter{Anhang}
\label{anhang}

Grundsätzlich sollte man Anhänge genau wie Fußnoten sehr sparsam und gezielt verwenden.
Der Text sollte ohne Anhang verständlich und schlüssig sein.
Der Anhang dient dazu, nicht notwendige, aber vielleicht trotzdem interessante Zusatzinformationen zu liefern.
Beispiel: ein Beweis für einen mathematischen Satz oder zusätzliches Bildmaterial, das die Zahlenergebnisse im Text untermauert, oder das vollständige UML-Diagramm für den Aufbau der Software.

Bitte fügen Sie keinen Source-Code als Anhang ein.

\section{Fußnoten und Abkürzungen}

Bei Fußnoten, ähnlich wie bei Anhängen, gibt es nur in Ausnahmefällen eine Rechtfertigung.
Bedenken Sie, dass der Blick auf eine Fußnote den Leser aus dem Textfluss reißt.
Er muss dann wieder den Satz finden, in dem auf die Fußnote verwiesen wurde. \Acp{fa} müssen einmal im Text ausgeschrieben werden und werden dann in Klammern hinter den Ausdruck geschrieben.
Ab dann kann die \ac{fa} verwendet werden. Fußnoten eignen sich hierfür nicht.
Bei anderen Erklärungen muss man sich fragen, ob die Erklärung wirklich wichtig und notwendig ist.
Dann spendieren Sie einfach einen eigenen Satz im Text und keine Fußnote.
Oder Sie kommen zu dem Schluss, dass die Erklärung eigentlich unwichtig ist, dann lassen Sie sie weg.

\section{Arbeitsorganisation}

Bei internen Abschlussarbeiten führen Sie ein Laborbuch oder Laborjournal und halten es im git-Repository (\code{doc-Verzeichnis}) aktuell.
Dabei halten Sie chronologisch fest, woran Sie gerade arbeiten, z.B. auch mit welchen Softwareversionen und Hilfsmitteln.
Das zwingt zu durchdachten Entscheidungen und man kann später nachblättern, wie es zu Richtungsentscheidungen gekommen ist.
Das Laborbuch wird \textbf{nicht kontrolliert} und ist \textbf{nicht Bestandteil der Notengebung}, sondern ein wichtiges Hilfsmittel für eine erfolgreiche Abschlussarbeit.
