
\appendix


\chapter{Anhang}
\label{anhang}

\paragraph{Hauptkapitel}
In diesem Dokument wird jedes Hauptkapitel auf der rechten Seite eines doppelseitigen Drucks begonnen.
Dadurch bleibt auch schon mal eine Seite leer, das macht aber nichts und erleichtert in der gebundenen Form das Lesen.
Im Allgemeinen sollten sich in irgendeiner Form die Kapitel \emph{ Einleitung}, \emph{Methode}, \emph{Ergebnisse}, \emph{Diskussion} und \emph{ Ausblick}.
Im Inhaltsverzeichnis sollten nur Kapitel bis zur Tiefe \emph{ Subsection}, also mit 3 Nummern, erscheinen.

Dieses Dokument ist als Hilfe gedacht, damit Sie schnell mit dem Tool \LaTeX~zurecht kommen,
damit Sie die wichtigsten Elemente einer sehr guten Abschlussarbeit zum Nachlesen haben und damit Sie meine \textbf{notenrelevanten Vorgaben} in schriftlicher Form vorliegen haben.
Mit diesem Template sollen nicht alle Abschlussarbeiten völlig gleich aussehen, insbesondere nicht sprachlich.
Entwickeln Sie Ihren eigenen Stil, seien Sie kreativ und bleiben Sie sich treu. Innerhalb dieser Vorlage gibt es jede Menge Spielraum.


\section{Länge}
\label{laenge}

Für Bachelor- und Masterarbeiten gilt, dass sie grundsätzlich so lang wie nötig und so kurz wie möglich sein sollten.
Bitte schinden Sie keine Seiten, wenn Sie nichts zu sagen haben.
Kommen Sie auf den Punkt und arbeiten sich anhand eines klaren roten Fadens von der Einleitung über einen Methodenteil zu Ergebnissen, Diskussion und Ausblick vor.
Gleichzeitig ist die Abschlussarbeit aber auch eine der möglicherweise seltenen Gelegenheiten, eine längere Abhandlung zu verfassen.
Dabei treten evtl.~Schwierigkeiten und Probleme auf, weil Sie das nicht gewöhnt sind.
Einmal vollbracht können Sie aber auch stolz auf Ihre Arbeit verweisen und Sie haben der Oma schon etwas zu Weihnachten zu schenken.

Dieser Lerneffekt stellt sich aber erst ein, wenn der Text auch eine kritische Länge übersteigt.
Deshalb gilt bei Abschlussarbeiten des ReMIC folgendes:
\begin{itemize}
\item Bachelorarbeiten: mindestens \textbf{45 Seiten} plus Titel, Inhaltsverzeichnis und Literaturverzeichnis
\item Masterarbeiten: mindestens \textbf{60 Seiten} plus Titel und Literaturverzeichnis
\end{itemize}
Sei $l_t$ die Länge des eigentlichen Textes inklusiver notwendiger Leerseiten, damit neue Kapitel immer rechts beginnen.
Sei weiter $l_i$ die Länge des Inhaltsverzeichnisses und  $l_l$ die Länge des Literaturverzeichnisses, dann ergibt sich die Gesamtlänge $g$ aus:
\begin{equation}
g = l_t + l_i + l_l + 1.
\label{gleichung laenge}
\end{equation}

\subsection{Meine Arbeit ist zu kurz. Was tun?}\label{zuKurz}

Sollten Sie Schwierigkeiten haben, $l_t \ge 45$ bzw. $l_t \ge 60$ aus (\ref{gleichung laenge}) zu erreichen, vergrößern Sie nicht die Graphiken unangemessen oder schinden ansonsten Seiten.
Denken Sie stattdessen über folgende Fragen nach:
\begin{enumerate}
\item Habe ich die Literatur zu meinem Thema ausreichend gewürdigt?
\item Habe ich eine Hinleitung zum Kernthema so geschrieben, dass meine Studienkolleginnen und -kollegen -- ohne Wikipedia zu bemühen -- folgen können?
\item Habe ich die Kernpunkte detailliert genug geschildert?
\item Habe ich alle sinnvollen Experimente tatsächlich durchgeführt, eine Hypothese daran verifiziert oder falsifiziert, das jeweilige Ergebnis nachvollziehbar beschrieben und diskutiert?
\item Habe ich über den Tellerrand hinausgesehen und in einem ausführlichen Ausblick künftige sinnvolle Schritte thematisiert?
\end{enumerate}

\section{Inhalt}
\label{inhalt}

Zielgruppe für Ihre Ausarbeitung ist nicht der betreuende Professor sondern Studierende der Informatik in ihrem letzten Semester.
Es ist wichtig, dass Sie in Ihrer Arbeit den Leser mitnehmen.
Dazu gehört, dass Sie einen klaren roten Faden verfolgen und sich nicht in Nebenkriegsschauplätzen verzetteln.
Gleichzeitig müssen Sie wissenschaftlich arbeiten.
\emph{Wissenschaftlich} heißt nicht, dass Sie in einer Bachelorarbeit Neues erfinden müssen.
Das ist bei einer Doktorarbeit so, abgeschwächt bei einer Masterarbeit auch, bei einer Bachelorarbeit nicht.
\emph{Wissenschaftlich} heißt vielmehr, sich seines eigenen Verstandes zu bedienen, die eigene Arbeit in den Kontext des bereits Vorhandenen zu stellen, Hypothesen aufzustellen und diese Hypothesen z.B. experimentell zu überprüfen.
Betrachten Sie Ihre Arbeit und Ihre Ergebnisse kritisch.
Machen Sie sich Gedanken, wie Sie die Korrektheit überprüfen können.
Geeignet können z.B. simulierte Daten sein, bei denen man das korrekte Ergebnis kennt.
Decken Sie mögliche Grenzfälle experimentell ab und verwenden nicht nur \emph{gutmütige} Beispiele.

\begin{table}[tb]\vspace{1ex}\centering
\begin{tabular*}{12cm}{ll|@{\extracolsep\fill}cccc}
&&\multicolumn{4}{c}{Verfahren} \\
&& A  & B &  C & D\\\hline
\multirow{5}*{\rotatebox{90}{Datensätze}}
& Patient 1 &  bla  & bla  & bla  & bla \\%\cline{2-6}
& Patient 2 & bla  & bla & bla  & bla  \\%\cline{2-6}
& Patient 3 &  bla  & bla & bla & bla \\%\cline{2-6}
& Patient 4 &  bla  & bla & \textbf{bestesBla} & bla \\%\cline{2-6}
& Patient 5 &  bla  & bla & bla & bla \\\hline
\end{tabular*}
\caption[Beispieltabelle]{Das ist ein Beispiel für eine recht komplexe Tabelle.
Nicht der gesamte Text der Tabellenunterschrift sollte im Tabellenverzeichnis auftauchen.
Hier wurde der beste Wert \textbf{fett} markiert.
\label{beispieltabelle}}
\vspace{2ex}\end{table}

Eine Arbeit wird immer aufgewertet durch objektive Ergebnisse in Form von Zahlen und Fakten.
Stellen Sie diese Zahlen geeignet dar, z.B. als Tabelle wie in Tab.~\ref{beispieltabelle} gezeigt.
Gut verständlich können auch Plots oder Diagramme sein.
Tabellen sollten horizontale und vertikale Trennlinien nur sparsam verwenden.


\section{Zitierung}\label{zitierung}

Sie sind verpflichtet, alle Quellen und Hilfmittel anzugeben, die Sie verwendet haben.
Dabei ist die wörtliche Zitierung in der Informatik selten; schon deshalb, weil viele Quellen englischsprachig sind und ein englisches wörtliches Zitat in einer deutschen Arbeit keinen Sinn macht.
Meist werden Gedanken, Ideen und Methoden aus Quellen entnommen.
Das Zitat kommt dann an den Schluß des letzten Satzes des Abschnittes, der z.B. die Methode beschreibt.
Bitte beachten Sie, dass das Zitat \textbf{vor} dem abschlißenden Punkt erscheint und nicht danach!

Alternativ wird zu Beginn oder am Ende der Beschreibung der Methode die Quelle mit Hilfe einer Phrase wie:
verwendet.

Als ein hilfreiches Werkzeug zum Auffinden von wissenschaftlichen Artikeln hat sich \url{http:\\\\scholar.google.com} erwiesen.

\paragraph{Zitierung von Internetseiten}
Grundsätzlich ist das Zitieren von Internetseiten so weit wie möglich zu vermeiden.
Internetseiten können plötzlich offline sein, der Inhalt kann sich ändern, manchmal ist der Autor unklar.
Genau das Gegenteil erwartet man von einer \emph{zitierfähigen} Quelle.
Dennoch läßt sich der Hinweis auf eine Internetseite manchmal nicht vermeiden.
In diesen Fällen versuchen Sie, den Autor herauszufinden und in die Quellenangabe einzufügen.
Wichtig ist der Hinweis auf das letzte Zugriffdatum.
Außerdem erwarte ich, dass Sie die Seite lokal speichern und den elektronischen Dokumenten auf der Abgabe-CD hinzufügen.

\section{Abgabe}
\label{abgabe}

Zur Abgabe beachten Sie bitte den Ablaufplan der Fakultät.
Dazu muss sind im Sekretariat spätestens zum offiziellen Abgabetermin zwei gebundene Exemplare Ihrer Arbeit abzugeben inklusive einer elektronischen Version zum Beispiel in Form einer CD.
Eine frühere Abgabe ist möglich.
Sollte das Sekretariat nicht besetzt sein, so werfen Sie Ihre beiden Exemplare bitte in das \textbf{dafür eingerichtete Postfach}, nicht in mein Postfach.
Ich selbst bekomme \textbf{kein} weiteres Exemplar, weil eines der beiden Exemplare sowie an mich weitergeleitet wird.
Auf der Abgabe-CD muss Folgendes enthalten sein:
\begin{enumerate}
\item PDF-Fassung Ihrer Ausarbeitung
\item Bildliche Zusammenfassung. Versuchen Sie, ein aussagekräftiges Bild zu erstellen, das Ihre Arbeit beschreibt und illustriert.
Es soll verwendet werden, um auf der ReMIC Webseite neben dem Titel der Arbeit auch das Bild zu zeigen, um eine schnelle Orientierung über Themen des ReMIC zu gewinnen.
\item Kopien der zitierten Internetseiten
\item Source-Code. Der Source-Code dient dazu, Ihren Programmierstil zu beurteilen. Dabei geht es z.B. auch darum, ob der Code aufgeräumt und gut dokumentiert ist.
\item Bilder. Die Bilder, die in der der Arbeit abgedruckt sind, möchte ich alle im Original als elektronische Variante vorfinden.
Aber die Bilder gehören ja sowieso ins git-Repository.
\end{enumerate}
Bitte beachten Sie, dass sich das Fehlen von Teilen dieser Liste nachteilig auf die Note auswirkt.

\paragraph{Übergabe} Bei internen ReMIC Arbeiten erwarte ich, dass die Lauffähigkeit der Software auf den Laborrechnern bei einem Übergabe-Termin nachgewiesen wird.
Basis dazu ist das git-Repository des Projektes.
Im ReadMe ist  genau zu beschreiben, was getan werden muss, um die Software zu kompilieren, welche Versionen von externen Libraries zum Test herangezogen wurden, etc.

Die Übergabe ist essentiell für die Nachhaltigkeit der Softwareentwicklung im ReMIC.
Der Übergabetermin sollte zeitnah {\emph{nach} der Abgabe entweder an eine Labormitarbeiterin oder einen Labormitarbeiter, einen wissenschaftlichen Mitarbeiter oder direkt an mich erfolgen.

% \paragraph{Code}
% Source Code kann durchaus sparsam dosiert in der Arbeit auftauchen.
% Aber nicht vollständig und als Anhang (siehe Anhang \ref{anhang}), sondern in sinnvollen Ausschnitten, wenn anhand des Code-Fragments ein Sachverhalt erläutert werden soll.
%
% \begin{lstlisting}[language=C++, breaklines=true, basicstyle=\small, numbers=none]
% // Das ist ein Beispiel fuer ein Codefragment
% int a = 7;
% int b = 3;
% int c = 10;
% a *= 2;
% c -= b + a++;
% std::cout << "c = " << c << std::endl;
% \end{lstlisting}
